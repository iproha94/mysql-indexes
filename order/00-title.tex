\newcommand{\HRule}{\rule{\linewidth}{0.5mm}}

\begin{center}

\textsc{Министерство образования и науки Российской Федерации
Федеральное государственное бюджетное образовательное учреждение 
высшего профессионального образования}\\[0.5cm]

\textsc{\large Московский государственный технический университет имени Н.\,Э.~Баумана}\\
\textsc{(МГТУ им. Н.Э.Баумана)}\\[0.5cm]
\includegraphics[scale=0.5]{bmstu-logo.png}~\\

\textsc{Факультет <<Информатика и системы управления>>}\\
\textsc{Кафедра <<Программное обеспечение ЭВМ\\и информационные технологии>>}\\[1cm]

% \HRule \\[0.5cm]
% {\huge \bfseries АЛГОРИТМ ОПРЕДЕЛЕНИЯ ВОЗМОЖНЫХ ИНДЕКСОВ, НЕОБХОДИМЫХ ДЛЯ ОПТИМИЗАЦИИ ЗАПРОСОВ С СОЕДИНЕНИЕМ ДВУХ ТАБЛИЦ В СУБД MYSQL (INNODB)}
% \HRule \\[0.5cm]

% \vfill

% \begin{flushright}
%   \begin{tabular}{rrlc}
%     Исполнитель:  &    студент ИУ7-81 & Петухов~И.\,С.  & \underline{\hspace{3cm}} \\[1cm]
%     Руководитель: & преподаватель ИУ7 & Просуков~Е.\,А. & \underline{\hspace{3cm}} \\[1cm]
%   \end{tabular}
% \end{flushright}

\textsc{\large Отчет по преддипломной практике}\\
\textsc{Тема практики}

\HRule 
{\bfseries \\ОПРЕДЕЛЕНИЕ ВОЗМОЖНЫХ ИНДЕКСОВ, НЕОБХОДИМЫХ ДЛЯ ОПТИМИЗАЦИИ ЗАПРОСОВ С СОЕДИНЕНИЕМ ДВУХ ТАБЛИЦ В СУБД MYSQL (INNODB)}
\HRule

\vfill

\begin{flushright}
  \begin{tabular}{rlc}
    Студент:  & Петухов~Илья\,Сергеевич  & \\[1cm]
    Группа: & ИУ7-81 & \\[1cm]
    Научный руководитель: & Просуков~Е.\,А. & \underline{\hspace{3cm}} \\[1cm]
    Руководитель практики: & Толпинская~Н.\,Б. & \underline{\hspace{3cm}} \\[1cm]
  \end{tabular}
\end{flushright}

{\large Москва, \the\year}

\end{center}

\pagenumbering{gobble}
\newpage
\pagenumbering{arabic}