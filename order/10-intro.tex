\Introduction

В работе рассматривается решение прикладной задачи, возникающей при работе с базами данных, которые уже спроектированы, используются в эксплуатации и имеют таблицы с несколькими тысячами записей. Проблема оптимизации работы базы данных решается с помощью индексирования. Рассматривается структура хранения данных в СУБД MySQL (InnoDB). Описываются кластерные и вторичные индексы. Для SQL запроса, имеющего операторы JOIN (INNER, LEFT, RIGHT), WHERE, ORDER BY предлагается два алгоритма. Первый алгоритм определяет, по какой из двух таблиц оператора JOIN будет осуществлено полное сканирование. Второй алгоритм показывает, какие индексы необходимо построить для оптимизации выполнения этого запроса. Оба алгоритма могут иметь программную реализацию. На примерах показывается использование этих алгоритмов. 