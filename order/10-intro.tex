\Introduction

Одной из главных причин низкой скорости обработки запросов является работа с базой данных (БД), которая не оптимизирована: нерациональные запросы, некорректное использование индексов, неоптимальные значения параметров конфигурации. 

Оптимизация производительности сводится к следующим задачам: 
\begin{enumerate}
\item корректировка параметров СУБД;
\item денормализация данных;
\item выявление медленных запросов и их анализ:
    \begin{enumerate}
    \item корректировка запросов;
    \item создание индексов.
    \end{enumerate}
\end{enumerate}

Основным приемом увеличения производительности выполнения запросов к базе данных является индексирование. Индексы представляют собой структуры данных, которые помогают СУБД эффективно извлекать данные. Они критичны для достижения хорошей производительности, но многие часто забывают о них или плохо понимают их смысл, поэтому индексирование является главной причиной проблем с производительностью в реальных условиях. \cite{zaitsev}

СУБД используют индексы для увеличения производительности выполнения SQL запросов путем оптимизации обращений к дисковой памяти (индексы в основном находятся в памяти).

Индексы следует создавать по мере обнаружения медленных запросов. В этом может помочь лог медленных запросов. Запросы, которые выполняются более 1 секунды, являются первыми кандидатами на оптимизацию. \cite{ruhighload-mysql-indexes} Конечно, говорить о том, сколько секунд считать медленным запросом зависит от конкретной задачи. В основном такие запросы используют JOIN оператор для соединения двух и более таблиц.

СУБД решают задачу выбора индекса для выполнения конкретного SQL запроса на работающей БД. Чтобы понять, какой индекс необходимо построить для оптимизации выполнения SQL запроса администратору БД необходимо учитывать:
\begin{enumerate}
    \item реальные данные, которыми заполнена БД
    \item статистику запросов;
    \item как СУБД применяет индексы;
    \item как СУБД соединяет таблицы (для сложных запросов).
\end{enumerate}

Администратору БД необходимо учитывать эти факторы, чтобы построить индекс для SQL запроса.

Если абстрагироваться от реальных данных и не учитывать статистику запросов, то зная как СУБД применяет индекса можно сделать предположения, какие индексы могли бы быть использованы для конкретного запроса. 

В данной работе предлагается алгоритм, возвращающий список индексов, которые СУБД могла бы использовать при выполнении этого запроса. Программа, реализующая предложенный алгоритм, облегчит работу администраторов БД, которым останется только принять решение, какой именно индекс построить из предложенного списка возможных индексов.
