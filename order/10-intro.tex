\Introduction

Рост количества хранимых данных и уникальных пользователей вынуждает разработчиков искать пути обеспечения высокого качества функционирования сервисов и низкого времени отклика пользовательского интерфейса. Одной из главных причин низкой скорости обработки запросов является работа с \nom{БД}{База данных}, которая не оптимизирована.  \cite{article:tsiganov}

Для \nom{АБД}{Администратор базы данных} и разработчиков приложений настройка приложений является критически важной задачей, и они тратят значительное время на ее выполнение. Плохо настроенное бизнес-приложение может потенциально повлиять не только на нескольких пользователей, но и на всю операционную деятельность организации, поэтому компании вкладывают значительные ресурсы, чтобы обеспечить хорошую работу приложений, критически важных для их бизнеса.\cite{fors.ru:Oracle-Database}

Ручная настройка SQL предоставляет собой очень сложный и проблемный процесс. Он требует глубокого знания в различных областях, отнимает много времени, требует детального знания структуры данных и модели использования данных приложения. Все эти факторы делают процесс ручной настройки SQL сложной и ресурсоемкой задачей, которая в конечном счете обходится очень дорого для бизнеса. \cite{fors.ru:Oracle-Database}

Оптимизация производительности сводится к следующим задачам: 
\begin{enumerate}
\item корректировка параметров СУБД;
\item денормализация данных;
\item выявление медленных запросов и их анализ:
    \begin{enumerate}
    \item корректировка запросов;
    \item создание индексов.
    \end{enumerate}
\end{enumerate}

Каждый из этих пунктов может стать темой отдельного исследования на определенных наборах данных, однако среднестатистическому разработчику необходимы рецепты, позволяющие быстро обойти проблемы производительности без досконального изучения документации по \nom{СУБД}{Система управления базой данных} и сосредоточить свое внимание на бизнес логике приложения. \cite{article:tsiganov}

Основным приемом увеличения производительности выполнения запросов к базе данных является индексирование. Индексы представляют собой структуры данных, которые помогают СУБД эффективно извлекать данные. Они критичны для достижения хорошей производительности, но многие часто забывают о них или плохо понимают их смысл, поэтому индексирование является главной причиной проблем с производительностью в реальных условиях. \cite{zaitsev}

В данной работе проводится обзор самых популярных СУБД и существующих для них инструментов для администрирование, а в частности индексирования.  Целью работы является разработка инструмента для облегчение работы администраторов СУБД MySQL по созданию индексов для SQL запросов.

Для достижения поставленной цели для выбранной СУБД необходимо изучить: 
\begin{enumerate}
\item доступные инструменты для анализа выполнения запросов
\item структуру хранения данных
\item типы используемых индексов
\item правила использования индексов
\item работу оптимизатора запросов
\end{enumerate}

На основе этой информации разработать инструмент, пригодный для использования АБД. Для этого нужно продумать и реализовать взаимодействие с пользователем.

Также необходимо предусмотреть возможность дальнейшей автоматизации процесса индексирования БД.