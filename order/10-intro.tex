% \Introduction
\chapter{Реферат}

В работе рассматривается решение проблемы оптимизации работы баз данных, которые уже спроектированы, используются в эксплуатации и имеют таблицы с несколькими тысячами записей. Данная проблема решается с помощью индексирования. 

Рассматривается структура хранения данных в СУБД MySQL (InnoDB). Описываются кластерные, вторичные, составные, покрывающие индексы. Рассмотрены правила применения индексов для запросов по одной таблице. 

Для SQL запроса с соединением двух таблиц предлагается два алгоритма. Первый алгоритм определяет, по какой из двух таблиц оператора JOIN будет осуществлено полное сканирование. Второй алгоритм показывает, какие индексы необходимо построить для оптимизации выполнения этого запроса. На примерах показывается использование этих алгоритмов. 

Для обоих алгоритмов создана программная реализация.