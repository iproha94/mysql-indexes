\section{MS SQL Server}

Данный программный продукт представляет собой СУБД реляционного типа, разработанную корпорацией Microsoft. Для манипуляции данными используется специально разработанный язык Transact-SQL. Команды языка для выборки и модификации базы данных построены на основе структурированных запросов.

СУБД является частью длинной цепочки специализированного программного обеспечения, которое корпорация Microsoft создала для разработчиков. А это значит, что все звенья этой цепи (приложения) глубоко интегрированы между собой. То есть их инструментарий легко взаимодействует между собой, что во многом упрощает процесс разработки и написания программного кода. Примером такой взаимосвязи является среда программирования MS Visual Studio. В ее инсталляционный пакет уже входит SQL Server Express Edition. \cite{internet-technologies.ru:SQL-Server}


\paragraph{Database Engine Tuning Advisor}

Помощник Database Engine Tuning Advisor анализирует рабочую нагрузку и выдает рекомендации по физической структуре одной или нескольких баз данных. Анализ содержит рекомендации по добавлению, удалению или модификации физических структур баз данных, таких как индексы, индексированные представления или секции. Помощник Database Engine Tuning Advisor рекомендует набор физических структур базы данных, которые оптимизируют задачи, входящие в рабочую нагрузку.

Рекомендации помощника Database Engine Tuning Advisor, связанные с физическими структурами, можно также просматривать, используя ряд отчетов, которые предоставляют информацию о некоторых представляющих значительный интерес опциях. Эти отчеты позволяют увидеть, как помощник Database Engine Tuning Advisor выполнял оценку рабочей нагрузки. Для просмотра доступны следующие отчеты:

\begin{enumerate}
\item Index Usage Report (recommended configuration (отчет по использованию индексов (рекомендуемая конфигурация)) — содержит информацию об ожидаемом использовании рекомендуемых индексов и их предполагаемых размерах;
\item Index Usage Report (current configuration) (отчет по использованию индексов (текущая конфигурация)) — предоставляет ту же самую информацию, что и предшествующий отчет, но для текущей конфигурации;
\item  Index Detail Report (recommended configuration) (подробный отчет по индексам (рекомендуемая конфигурация)) — содержит информацию об именах всех рекомендуемых индексов и их типах;
\item Index Detail Report (current configuration) (подробный отчет по индексам (текущая конфигурация)) — предоставляет ту же самую информацию, что и предшествующий отчет, но для фактической конфигурации до начала процесса настройки;
\item Table Access Report (отчет о доступе к таблицам) — предоставляет информацию о затратах всех запросов в рабочей нагрузке (используя таблицы базы данных);
\item Workload Analysis Report (отчет анализа рабочей нагрузки) — предоставляет информацию об относительных частотах всех инструкций по модификации данных (затраты подсчитываются относительно наиболее затратной инструкции при текущей конфигурации индексов).
\end{enumerate}

Эти рекомендации можно использовать тремя способами: немедленно, по расписанию или после сохранения в файл. \cite{petkovich:sql-server-2012}

\section{Azure SQL Database}

\textit{SQL Azure} – проекция традиционного SQL Server на облако, предоставляющая возможности для работы с базой данных посредством интернет-сервисов. Эта технология позволяет хранить структурированную и неструктурированную информацию, исполнять реляционные запросы, а также предоставляет функционал для осуществления поиска, создания аналитических отчётов, интеграции и синхронизации данных. На данный момент SQL Azure поддерживает сервис реляционных баз данных, имеющий название SQL Azure Database. 

\textit{SQL Azure Database} – облачная платформа реляционной базы данных, построенная на технологиях SQL Server. При использовании этой платформы можно легко построить в облаке проект реляционной базы данных со всеми преимуществами, предоставляемыми любой облачной технологией. Кроме того, SQL Azure предоставляет высокий уровень безопасности со встроенной защитой данных, самовосстановлением и системой резервного копирования. Хотя SQL Azure и базируется на технологиях SQL Server, он представляет такие новые возможности, как высокий уровень масштабируемости, постоянная доступность и самоуправление, предоставляя клиентам легкие и удобные способы работы посредством сети Интернет, не требуя при этом особенных навыков или знаний, отличных от применимых с технологиями традиционного SQL Server. \cite{habrahabr.ru:sql-azure}


\paragraph{Index Advisor}

Используя помощник по базам данных SQL Azure на портале Azure можно просматривать и реализовывать рекомендации для существующих баз данных SQL, которые могут повысить текущую производительность запросов. На странице рекомендаций приводится список основных предлагаемых рекомендаций с учетом их потенциального влияния на повышение производительности. 

Помощник по работе с базами данных SQL можно настроить на автоматическое выполнение рекомендаций. В этом случае появляющиеся рекомендации будут применяться автоматически. Как и во всех остальных операциях с индексами, управляемых службой, если выполнение рекомендации ведет к ухудшению производительности, она отменяется.

Помощник по работе с базами данных SQL предоставляет рекомендации по повышению производительности базы данных SQL. Предоставляя сценарии T-SQL, а также параметры индивидуального или полностью автоматизированного управления (в настоящее время только для индексов), помощник помогает оптимизировать базу данных и, таким образом, повысить производительность запросов. \cite{docs.microsoft.com:helper-azure}

Index Advisor поможет найти недостающие индексы (а так же предложит удалить ненужные) и улучшить производительность базы данных.
