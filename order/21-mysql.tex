\subsection{MySQL}

Одной из самых распространенных реляционных СУБД является MySQL. Она кроссплатформенна, свободно распространяется по лицензии GNU и такими компаниями как Google, Adobe, Facebook и др. \cite{article:tsiganov}

MySQL является решением для малых и средних приложений. Входит в состав серверов WAMP, AppServ, LAMP и в портативные сборки серверов Денвер, XAMPP, VertrigoServ. Обычно MySQL используется в качестве сервера, к которому обращаются локальные или удалённые клиенты, однако в дистрибутив входит библиотека внутреннего сервера, позволяющая включать MySQL в автономные программы. 

MySQL портирована на большое количество платформ: AIX, BSDi, FreeBSD, HP-UX, Linux, Mac OS X, NetBSD, OpenBSD, OS/2 Warp, SGI IRIX, Solaris, SunOS, SCO OpenServer, UnixWare, Tru64, Windows 95, Windows 98, Windows NT, Windows 2000, Windows XP, Windows Server 2003, WinCE, Windows Vista, Windows 7 и Windows 10. Существует также порт MySQL к OpenVMS. Важно отметить, что на официальном сайте СУБД для свободной загрузки предоставляются не только исходные коды, но и откомпилированные и оптимизированные под конкретные операционные системы готовые исполняемые модули СУБД MySQL.\cite{wikipedia.org:mysql}



\paragraph{EXPLAIN}

Если оператор SELECT предваряется ключевым словом EXPLAIN, MySQL сообщит о том, как будет производиться обработка SELECT, и предоставит информацию о порядке и методе связывания таблиц.

При помощи EXPLAIN можно выяснить, когда стоит снабдить таблицы индексами, чтобы получить более быструю выборку, использующую индексы для поиска записей. Кроме того, можно проверить, насколько удачный порядок связывания таблиц был выбран оптимизатором. 

Для непростых соединений EXPLAIN возвращает строку информации о каждой из использованных в работе оператора SELECT таблиц. Таблицы перечисляются в том порядке, в котором они будут считываться. \cite{mysql.ru:EXPLAIN}

\paragraph{Профайлер}

Директива \textit{SHOW PROFILES} показывает список запросов, выполненных в рамках текущей сессии и время выполнения каждого запроса. Директива \textit{SHOW PROFILE} показывает подробную информацию об этапах выполнения
отдельного запроса. По умолчанию, выводится информация о последнем запросе.

\paragraph{Percona Toolkit}

Штатные инструменты поставляемые с MySQL предоставляют лишь базовые возможности по администрированию, в результате многие операции приходится выполнять вручную. Это может быть проблемой, ведь уследить за всем очень сложно и часто потребуется определенный опыт, да и легко допустить ошибку. Пакет Percona Toolkit for MySQL собрал наработки двух проектов Maatkit и Aspersa и предоставляет скрипты позволяющие производить многие рутинные операции администрирования: \cite{xakep.ru:mysql-admin-toolkit, blog.dh.md:Percona_Toolkit}

\begin{enumerate}
\item pt-archiver - архивирует записи из одной таблицы в другую, можно задать условие
\item pt-deadlock-logger - логирует информацию о мертвых блокировках 
\item pt-diskstats - мониторинг загрузки дисков
\item pt-duplicate-key-checker - находит дубликаты индексов в базе
\item pt-fk-error-logger - логирует информацию об ошибках внешних ключей
\item pt-heartbeat - мониторинг задержек репликации
\item pt-index-usage - читает запросы из логов и анализирует как используются индексы
\item pt-kill - убивает запросы, подходящие под те или иные условия.
\item pt-mysql-summary - суммарная информация о сервере MySQL.
\item pt-online-schema-change - изменение таблицы без блокировки - создает пустую таблицу - делает все изменения, затем переносит данные в новую таблицу и переименовывает ее.
\item pt-query-digest - анализирует запросы из slow.log или из processlist
\item pt-show-grants - показывает доступы всех пользователей.
\item pt-table-usage - анализирует запросы из лога и как они использую таблицы.
\item pt-variable-advisor - анализирует переменные MySQL и выдает советы по возможным проблемам
\item pt-visual-explain - форматирует вывод EXPLAIN в виде дерева
\end{enumerate}