\section{MySQL}

Одной из самых распространенных реляционных СУБД является MySQL. Она кроссплатформенна, свободно распространяется по лицензии GNU и такими компаниями как Google, Adobe, Facebook и др. \cite{article:tsiganov}

MySQL является решением для малых и средних приложений. Входит в состав серверов WAMP, AppServ, LAMP и в портативные сборки серверов Денвер, XAMPP, VertrigoServ. Обычно MySQL используется в качестве сервера, к которому обращаются локальные или удалённые клиенты, однако в дистрибутив входит библиотека внутреннего сервера, позволяющая включать MySQL в автономные программы. 

MySQL портирована на большое количество платформ: AIX, BSDi, FreeBSD, HP-UX, Linux, Mac OS X, NetBSD, OpenBSD, OS/2 Warp, SGI IRIX, Solaris, SunOS, SCO OpenServer, UnixWare, Tru64, Windows 95, Windows 98, Windows NT, Windows 2000, Windows XP, Windows Server 2003, WinCE, Windows Vista, Windows 7 и Windows 10. Существует также порт MySQL к OpenVMS. Важно отметить, что на официальном сайте СУБД для свободной загрузки предоставляются не только исходные коды, но и откомпилированные и оптимизированные под конкретные операционные системы готовые исполняемые модули СУБД MySQL.\cite{wikipedia.org:mysql}



\paragraph{EXPLAIN}

Если оператор SELECT предваряется ключевым словом EXPLAIN, MySQL сообщит о том, как будет производиться обработка SELECT, и предоставит информацию о порядке и методе связывания таблиц.

При помощи EXPLAIN можно выяснить, когда стоит снабдить таблицы индексами, чтобы получить более быструю выборку, использующую индексы для поиска записей. Кроме того, можно проверить, насколько удачный порядок связывания таблиц был выбран оптимизатором. 

Для непростых соединений EXPLAIN возвращает строку информации о каждой из использованных в работе оператора SELECT таблиц. Таблицы перечисляются в том порядке, в котором они будут считываться. \cite{mysql.ru:EXPLAIN}

\paragraph{Профайлер}

Директива \textit{SHOW PROFILES} показывает список запросов, выполненных в рамках текущей сессии и время выполнения каждого запроса. Директива \textit{SHOW PROFILE} показывает подробную информацию об этапах выполнения
отдельного запроса. По умолчанию, выводится информация о последнем запросе.

\paragraph{Percona Toolkit}

Штатные инструменты поставляемые с MySQL предоставляют лишь базовые возможности по администрированию, в результате многие операции приходится выполнять вручную. Это может быть проблемой, ведь уследить за всем очень сложно и часто потребуется определенный опыт, да и легко допустить ошибку. Пакет Percona Toolkit for MySQL собрал наработки двух проектов Maatkit и Aspersa и предоставляет скрипты позволяющие производить многие рутинные операции администрирования: \cite{xakep.ru:mysql-admin-toolkit}

\begin{enumerate}
\item pt-archiver - архивирует записи из одной таблицы в другую, можно задать условие
\item pt-deadlock-logger - логирует информацию о мертвых блокировках 
\item pt-diskstats - мониторинг загрузки дисков
\item pt-duplicate-key-checker - находит дубликаты индексов в базе
\item pt-fk-error-logger - логирует информацию об ошибках внешних ключей
\item pt-heartbeat - мониторинг задержек репликации
\item pt-index-usage - читает запросы из логов и анализирует как используются индексы
\item pt-kill - убивает запросы, подходящие под те или иные условия.
\item pt-mysql-summary - суммарная информация о сервере MySQL.
\item pt-online-schema-change - изменение таблицы без блокировки - создает пустую таблицу - делает все изменения, затем переносит данные в новую таблицу и переименовывает ее.
\item pt-query-digest - анализирует запросы из slow.log или из processlist
\item pt-show-grants - показывает доступы всех пользователей.
\item pt-table-usage - анализирует запросы из лога и как они использую таблицы.
\item pt-variable-advisor - анализирует переменные MySQL и выдает советы по возможным проблемам
\item pt-visual-explain - форматирует вывод EXPLAIN в виде дерева
\end{enumerate}
\cite{blog.dh.md:Percona_Toolkit}


% Рассмотрим еще некоторые популярные инструменты, упрощающие администрирование MySQL.

% \paragraph{openark kit}

% Набор Openark предоставляет общие утилиты для администрирования, диагностики и аудита баз данных MySQL.

% Предлагает 14 утилит, позволяющих провести тестирование СУБД: проверять установки, проверять пароли (пустые, одинаковые, слабые), блокировать аккаунты, прерывать запросы, фильтровать записи в журнале, выводить статус репликации, исправлять кодировки и многое другое. Распространяется по лицензии BSD. Написан на Python. \cite{xakep.ru:mysql-admin-toolkit}

% \paragraph{Тюнинг MySQL}

% Оптимизация настроек очень тонкий процесс, ведь нужно на основании собранной статистики изменить только то, что действительно повлияет на производительность. Самым известным инструментом для MySQL является Perl-скрипт MySQLTuner, который доступен в репозиториях большинства дистрибутивов Linux. Он читает текущие настройки сервера и установки MySQL, после чего выдает рекомендации (только рекомендации) по их изменению. \cite{xakep.ru:mysql-admin-toolkit}

% \paragraph{Workbench}

% MySQL Workbench распространяется под свободной лицензией — Community Edition и с ежегодной оплачиваемой подпиской — Standard Edition. Последняя включает в себя дополнительные возможности, которые способны существенно улучшить производительность, как разработчиков, так и администраторов баз данных. \cite{habrahabr.ru:10_best_tools}

% \begin{enumerate}
% \item возможность представить модель БД в графическом виде, а также редактирование данных в таблице;
% \item наличие простого и функционального механизма по созданию связей между полями таблиц, среди которых реализована связь «многие-ко-многим» с возможностью создания таблицы связей;
% \item функция Reverse Engineering позволяет восстанавливать структуру таблиц и связей из той, которая была реализована ранее и хранится на сервере БД;
% \item наличие редактора SQL-запросов, который дает возможность при отправке на сервер получать ответ в табличном виде и другие возможности.
% \end{enumerate}

% \paragraph{Navicat}

% Navicat (разработка компании PremiumSoft CyberTech Ltd) — инструмент для разработки и администрирования баз данных, который работает на любом сервере MySQL, начиная с версии 3.21. Для MySQL, Navicat доступен для работы на платформах Microsoft Windows, Mac OS X и Linux. \cite{habrahabr.ru:10_best_tools}

% \begin{enumerate}
% \item наличие визуального конструктора запросов;
% \item возможность импорта, экспорта и резервного копирования данных;
% \item возможность создавать отчеты;
% \item SSH и HTTP туннелинг;
% \item миграция и синхронизация данных и структуры;
% \item инструмент для планирования задач и другие возможности.
% \end{enumerate}

% \paragraph{PHPMyAdmin}

% PHPMyAdmin — веб-приложение с открытым кодом, написанное на языке PHP и представляющее собой веб-интерфейс для администрирования СУБД MySQL. PHPMyAdmin позволяет через браузер и не только осуществлять администрирование сервера MySQL, запускать команды SQL и просматривать содержимое таблиц и баз данных. Приложение пользуется большой популярностью у веб-разработчиков, так как позволяет управлять СУБД MySQL без непосредственного ввода SQL команд, предоставляя дружественный интерфейс.

% На сегодняшний день PHPMyAdmin широко применяется на практике. Последнее связано с тем, что разработчики интенсивно развивают свой продукт, учитывая все нововведения СУБД MySQL. Подавляющее большинство российских провайдеров используют это приложение в качестве панели управления для того, чтобы предоставить своим клиентам возможность администрирования выделенных им баз данных. \cite{wikipedia.org:phpmyadmin}


% \paragraph{dbForge Studio for MySQL}

% dbForge Studio for MySQL — универсальное решение для разработки, администрирования и управления базами данных MySQL и MariaDB. Данный продукт позволяет создавать и выполнять запросы, разрабатывать и отлаживать процедуры и функции, а также автоматизировать управление объектами баз данных MySQL с помощью удобного пользовательского интерфейса. dbForge Studio также содержит инструменты для сравнения, синхронизации, создания резервных копий баз данных по графику, а также для анализа и создания отчетов по данным таблиц MySQL. \cite{devart.com:dbforge}

% \begin{enumerate}
% \item Интеллектуальная разработка SQL кода
% \item Сравнение и синхронизация БД
% 	\begin{enumerate}
% 	\item Сравнивать и синхронизировать данные и схемы
% 	\item Планировать стандартные задачи по синхронизации БД
% 	\item Генерировать отчеты о сравнении
% 	\end{enumerate}
% \item Визуальный дизайнер запросов
% \item Дизайнер баз данных:
% 	\begin{enumerate}
% 	\item Просмотра связей по внешним ключам
% 	\item Отображения объектов БД со свойствами
% 	\item Выполнение хранимых процедур
% 	\end{enumerate}
% \item Импорт/экспорт данных
% \item Резервные копии БД
% \item Инструменты для администрирования и управления базами данных MySQL включают средства для:
% 	\begin{enumerate}
% 	\item Управления ролями и привилегиями пользователей
% 	\item Контроля сервисов MySQL
% 	\item Управления переменными сервера
% 	\item Обслуживания таблиц
% 	\item Управления сессиями
% 	\end{enumerate}
% \item Отладчик MySQL 
% \item Дизайнер таблиц
% \item Рефакторинг базы данных
% \item Профилировщик запросов
% \item Отчеты и анализ данных
% \end{enumerate}
