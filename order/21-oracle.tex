\subsection{Oracle}

Oracle Database – СУБД, ориентированная на применение в корпоративных сетях распределенной обработки данных (Enterprise Grid), в облачных системах (Cloud Computing), а также для построения корпоративных информационных систем. Она позволяет сократить расходы на информационные технологии благодаря автоматизации управления, использованию недорогих модульных компонентов и кластеризации серверов в целях эффективного использования ресурсов. 

Архитектура СУБД Oracle рассчитана на работу с огромными объемами данных и большим (десятки и сотни тысяч) числом пользователей; она демонстрирует широкие возможности обеспечения высокой готовности, производительности, масштабируемости, информационной безопасности и самоуправляемости. СУБД Oracle может быть развернута на любой платформе, начиная от небольших серверов-лезвий и заканчивая симметричными многопроцессорными компьютерами и мейнфреймами. Уникальная способность СУБД Oracle работать со всеми типами данных, от традиционных таблиц до XML-документов и картографических данных, позволяет рассматривать ее в качестве оптимального выбора для работы с приложениями оперативной обработки транзакций, поддержки принятия решений и управления коллективной работой с информацией.  

\paragraph{Oracle Tuning Pack} – дополнительная опция для управления Oracle Database, наиболее эффективное и легкое в использовании решение, которое полностью автоматизирует процесс настройки приложений. Улучшение производительности SQL достигается с помощью мониторинга выполнения SQL в реальном времени и SQL-советников, интегрированных с Oracle Enterprise Manager Cloud Control 12c, и все это вместе предоставляет всестороннее решение для сложной и требующей много времени задачи по настройке приложений.

\textit{SQL Tuning Advisor} является ответом Oracle на все недостатки и проблемы ручной настройки SQL. Он автоматизирует процесс настройки SQL путем всестороннего исследования всех возможных вариантов настройки SQL-предложения. Анализ и настройка осуществляются с помощью существенно улучшенного оптимизатора запросов, встроенного в ядро базы данных.

\textit{SQL Tuning Advisor} проводит шесть типов анализа:

\begin{enumerate}
\item Анализ статистики: выявление объектов с отсутствующей или устаревшей статистикой, выдача соответствующих рекомендаций по устранению проблемы.
\item SQL-профилирование: Эта возможность, появившаяся в Oracle Database 10g, революционизировала подход к настройке SQL. SQL-профилирование позволяет настраивать SQL-предложения без каких-либо изменений кода приложения.
\item Анализ путей доступа: Во время этого анализа определяются новые индексы, которые могут значительно улучшить производительность запросов.
\item Анализ структуры SQL: Здесь проверяется неявное преобразование типов и даются рекомендации по изменению кода SQL.
\item Степень параллелизма: SQL Tuning Advisor определяет, можно ли улучшить время выполнения с помощью параллельных потоков на определенных этапах выполнения SQL.
\item Альтернативные планы: Во время этого анализа SQL Tuning Advisor находит другие планы выполнения запроса, используя текущие и исторические данные производительности.
\end{enumerate}

Он всесторонне анализирует всю нагрузку и дает рекомендации по созданию новых секций таблицы или индексов, удалению неиспользуемых индексов, созданию новых материализованных представлений и журналов. Определение оптимальной стратегии секционирования или индексирования для конкретной нагрузки является сложным процессом, требующим опыта и времени. SQL Access Advisor учитывает стоимость операций ввода/обновления/удаления в дополнение к запросам и дает соответствующие рекомендации, сопровождаемые количественной мерой ожидаемого выигрыша в производительности, а также скрипты, необходимые для реализации этих рекомендаций. 

Настройка SQL-предложений больше не является прерогативой только специалистов. Oracle встроила эксперта по настройке SQL в ядро базы данных, позволив администраторам баз данных выполнять эту очень важную задачу за доли времени и затрат, необходимых для выполнения той же задачи вручную.

\cite{fors.ru:Oracle-Database}