\section{PostgreSQL}

PostgreSQL это мощная объектно-реляционная система управления базами данных с открытыми исходными текстами. Она разрабатывается на протяжении более 15 лет и улучшает архитектуру, чем завоевала репутацию надежной, ингерированной и масштабируемой СУБД. Она запускается на всех основных платформах, включая Linux, UNIX (AIX, BSD, HP-UX, SGI IRIX, Mac OS X, Solaris, Tru64), и Windows. Она полностью соответствует ACID, имеет полную поддержку ключей, объединений, представлений, триггеров, и хранимых процедур (на разных языках). Она включает большинство типов данных SQL92 и SQL99, включая INTEGER, NUMERIC, BOOLEAN, CHAR, VARCHAR, DATE, INTERVAL, и TIMESTAMP. Она также поддерживает хранение больших двоичных объектов (BLOB's), включая картинки, звук, или видео. Она имеет API для C/C++, Java, Perl, Python, Ruby, Tcl, ODBC. 

Являясь СУБД класса предприятия, PostgreSQL предоставляет такие особенности как Multi-Version Concurrency Control (MVCC), восстановление по точке во времени, табличное пространство, асинхронная репликация, вложенные транзакции (точки сохранения), горячее резервирование, планировщик/оптимизатор запросов, и упреждающее журналирование на случай поломки. Он поддерживает международные кодировки, в том числе и многобайтовые, при использование различных кодировок можно использовать сортировку и полнотекстовый поиск, различать регистр. Большое количество подконтрольных данных и большое число одновременно работающих пользователей, тем не менее, не сильно влияет на масштабируемость системы. Есть действующие PostgreSQL системы, которые управляют более чем 4 терабайтами данных. \cite{opensuse.org:Postgresql}

\paragraph{Index Advisor}

Утилита \textit{Index Advisor} помогает определить, какие столбцы следует индексировать, чтобы повысить производительность в заданной рабочей нагрузке. \textit{Index Advisor} рассматривает типы индексов B-tree (одностолбцовые или составные) и не идентифицирует другие типы индексов (GIN, GiST, Hash), которые могут повысить производительность. \textit{Index Advisor} устанавливается вместе с \textit{Postgres Plus Advanced Server}.

\textit{Index Advisor} работает с планировщиком запросов \textit{Advanced Server}, создавая гипотетические индексы, которые планировщик запросов использует для расчета затрат на выполнение, как если бы такие индексы были доступны. \textit{Index Advisor} определяет индексы, анализируя SQL-запросы, поставляемые в рабочей нагрузке.

Один из способов использования \textit{Index Advisor}  для анализа SQL-запросов, это вызов служебной программы, предоставив текстовый файл, содержащий SQL-запросы, которые необходимо проанализировать; \textit{Index Advisor} сгенерирует текстовый файл с инструкциями \textit{CREATE INDEX} для рекомендуемых индексов.

В ходе анализа \textit{Index Advisor} сравнивает затраты на выполнение запроса с гипотетическими индексами и без них. Если стоимость исполнения с использованием гипотетического индекса меньше стоимости исполнения без него, что сообщается в выводе инструкции EXPLAIN, где вычисляются показатели, которые оценивают улучшение, то \textit{Index Advisor} генерирует инструкцию \textit{CREATE INDEX}, необходимую для создания индекса.

\textit{Index Advisor} фактически не создает индексы в таблицах. необходимо использовать инструкции \textit{CREATE INDEX}, предоставляемые \textit{Index Advisor}, чтобы добавить в таблицы все рекомендуемые индексы.

\textit{Pg_advise_index} - это служебная программа, которая считывает предоставленный пользователем входной файл, содержащий SQL-запросы, и создает текстовый файл, содержащий инструкции \textit{CREATE INDEX}, которые можно использовать для создания индексов, рекомендованных \textit{Index Advisor}.


