\subsection{Индексы для WHERE}

Для запроса на листинге \ref{sql:index-on-where} можно построить индексы \textit{(dob, last\_name)} и \textit{(last_name, dob)}.
\begin{lstlisting}[language=sql, label=sql:index-on-where, caption={запрос для index-on-where}]
SELECT * 
FROM poet
WHERE last_name = ”Пушкин” 
    AND dob = 1799
\end{lstlisting}

Рассмотрим работу индексов на примере индекса \textit{(a, b, c)}, где \textit{a, b} - числа, \textit{c} - строка.

\paragraph{Примеры запросов, к которым применяется индекс \textit{(a, b, c)}}

\begin{enumerate}
\item \textit{a = 5 AND b = 10 AND c="Hello world"}
\item \textit{a = 5}
\item \textit{a > 5}  
\item \textit{a = 5 AND b = 10} 
\item \textit{a = 5 AND b = 10 AND c LIKE "Hello w\%"} 
\item \textit{a = 5 AND b IN (2,3)} (\textit{IN} - рассматривается как поиск по диапазону)
\end{enumerate}

\paragraph{Примеры запросов, к которым не применяется индекс \textit{(a, b, c)}}

\begin{enumerate}
\item \textit{b = 5}
\item \textit{a=5 AND b=10 AND c LIKE "\% world"}
\item \textit{a=5 AND c=10}
\item \textit{a>5 AND b=2}
\end{enumerate}

