\subsection{Индексы при сортировке}

Чтобы получить отсортированную последовательность данных, MySQL достаточно будет пройтись по дереву, т.к. в индексе данные хранятся в отсортированном виде. 

Для запроса на листинге \ref{sql:index-order1} строим индекс $(dob)$.
\begin{lstlisting}[language=sql, label=sql:index-order1, caption={запрос для index-order}]
SELECT * 
FROM poet
ORDER BY dob
\end{lstlisting}

Для запроса на листинге \ref{sql:index-order2} строим индекс $(country, dob)$. В индексе находим строки, удовлетворяющие условию $country="ru"$, а в этой выборке строки уже отсортированы по $dob$.
\begin{lstlisting}[language=sql, label=sql:index-order2, caption={запрос для index-order}]
SELECT * 
FROM poet
WHERE country="ru" 
ORDER BY dob
\end{lstlisting}

Для запроса на листинге \ref{sql:index-order3} строим индекс $(dob)$. $GROUP\:BY$ возьмет уже отсортированные строки из индекса и уберет повторы, а т.к. строки уже отсортированные - $ORDER\:BY$ всего лишь задаст, в каком порядке выводить данные. 
\begin{lstlisting}[language=sql, label=sql:index-order3, caption={запрос для index-order}]
SELECT *
FROM poet
GROUP BY dob
ORDER BY dob
\end{lstlisting}


\paragraph{Index-order правила}

Рассмотрим работу индексов на примере индекса $(a, b)$, где $a, b$ - числа.

Примеры работы индекса $(a, b)$:
\begin{enumerate}
\item сортировка по первой колонке
\item первая колонка в условии $WHERE$, и сортировка по второй колонке
\item первая колонка в условии $WHERE$ и сортировка по первой колонке
\item сортировка по двум колонкам и обе в одну сторону
\end{enumerate}

Примеры, когда индекс $(a, b)$ не работает:
\begin{enumerate}
\item $ORDER\:BY \; b$ (т.к. $b$ - не левый префикс индекса)
\item $WHERE a>5 \; ORDER\,BY\:B$ \\
Пример данных в индексе $(a, b)$: $a=6,\:b=2; \; a=6,\:b=3; \; a=7,\: b=0; \; a=8,\:b=1$.\\
MySQL сделает выборку по условию $a>5$, но в этой выборке строки не отсортированы по $b$.
\item $WHERE a\:IN\:(1,2) \; ORDER\:BY\:b$ (тоже самое, что и в предыдущем запросе)
\item сортировка разных столбцов в разных направлениях (чтобы обойти это, можно сделать виртуальную колонку, например, перед числом поставить минус)
\end{enumerate}