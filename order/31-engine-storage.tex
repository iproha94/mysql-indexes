\section{Подсистемы хранения MySQL}

При разработке приложения для MySQL необходимо решить, какую подсистему хранения использовать. Поскольку допустимо выбирать способ хранения данных для каждой таблицы в отдельности, нужно ясно понимать, как будет использоваться каждая таблица, и какие данные в ней планируется хранить. Это также поможет получить хорошее представление о приложении в целом и о потенциале его роста. Вооружившись этой информацией, можно осознанно выбирать подсистемы хранения данных.

Согласно \cite{tagline.ru:List_DBMS} самые распространенные подсистемы хранения: MyISAM — 74,4\%, InnoDB — 68,5\%, другие движки — 31\%. В связи с тем, что один и тот же тип индекса в разных подсистемах хранения может быть реализован по-разному, они будут иметь свои приемущества и недостатки. \cite[p.~137]{zaitsev}

\subsection{Подсистема MyISAM}

Как одна из самых старых подсистем хранения, включенных в MySQL, MyISAM обладает многими функциями, которые были разработаны за годы использования СУБД для решения различных задач. Она предоставляет полнотекстовое индексирование, сжатие и пространственные функции (для геоинформационных систем – ГИС). MyISAM не поддерживает транзакции и блокировки на уровне строк.


\subsection{Подсистема InnoDB}

Подсистема хранения InnoDB была разработана для транзакционной обработки, в частности для обработки большого количества краткосрочных транзакций, которые значительно чаще благополучно завершаются, чем откатываются. Она остается наиболее популярной транзакционной подсистемой хранения. Высокая производительность и автоматическое восстановление после сбоя делают ее популярной и для нетранзакционных применений.
