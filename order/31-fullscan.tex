\section{Fullscan алгоритм}

При соединении двух таблиц (A и B), для каждой строки одной таблицы A будет происходить сканирование другой таблицы B. Такое сканирование таблицы A назовем \textbf{полным сканированием}. 

\textbf{Fullscan алгоритм} - алгоритм для определения таблицы, по которой будет осуществлено полное сканирование. Алгоритм на вход принимает текст запроса и возвращает название таблицы, по которой будет полное сканирование или null, если таблица не определена (в таком случае MySQL после выбора подходящих индексов выберет, по какой таблице будет происходить полное сканирование).

В качестве fullscan алгоритма для join-запроса можно предложить алгоритм, при работе которого происходит анализ исследуемого запроса. Псевдокод представлена в алгоритме \ref{alg:fullscan}. Блок-схема представлена на рисунке \ref{img:fullscan}.

\begin{algorithm}[h]
\caption{Fullscan алгоритм}\label{alg:fullscan}
\begin{algorithmic}[1]
 
\Function{fullscanAlg}{query}
    \If {условие соединения $A LEFT JOIN B$}
        \If {в условии WHERE исключается возможность равенства любого из полей таблицы B значению null}
            \State {заменить $LEFT JOIN$ на $INNER JOIN$}
        \Else
            \State \Return таблица $A$
        \EndIf
    \EndIf
    \Statex 
    
    \If {условие соединения $A RIGHT JOIN B$}
        \If {в условии WHERE исключается возможность равенства любого из полей таблицы $A$ значению null}
            \State {заменить $RIGHT JOIN$ на $INNER JOIN$}
        \Else
            \State \Return таблица $B$
        \EndIf
    \EndIf
    \Statex 

    \If {условие соединения $A INNER JOIN B$}
        \If {есть сортировка $ORDER BY A.a, \ldots$}
            \State \Return таблица $A$
        \ElsIf {есть сортировка $ORDER BY B.a, \ldots$}
            \State \Return таблица $B$
        \EndIf
    \EndIf
    \Statex 

    \State \Return $null$
\EndFunction
 
\end{algorithmic}
\end{algorithm}

\begin{figure}[h]
  \centering
  \includegraphics[scale=0.5]{fullscan.png}
  \caption{Блок-схема fullscan-алгоритма.}
  \label{img:fullscan}
\end{figure}


\subsection{Примеры работы алгоритма}

Рассмотрим работу fullscan алгоритма по обработке заданного SQL запроса на конкретных практических примерах.

\paragraph{Пример №1}

\begin{lstlisting}[language=SQL]
query = {FROM t1 LEFT JOIN t2 ON t1.a = t2.a WHERE t1.b > 1 AND t2.c = 5}
\end{lstlisting}

Условие соединения - $LEFT JOIN$. В условии $WHERE$ исключается возможность равенства поля $t2.c$ значению $null$, значит изменим запрос на
\begin{lstlisting}[language=SQL]
query = {FROM t1 INNER JOIN t2 ON t1.a = t2.a WHERE t1.b > 1 AND t2.c = 5}
\end{lstlisting}

Условие соединения - $INNER JOIN$. Сортировок нет, значит вернуть $null$. 

\textbf{Ответ:} по какой таблице будет осуществлено полное сканирование на данном этапе не определено.


\paragraph{Пример №2}
\begin{lstlisting}[language=SQL]
query = {FROM t1 LEFT JOIN t2  ON t1.a = t2.a WHERE t1.b = 5000 AND t1.c > 3 ORDER BY t2.c, t2.d}
\end{lstlisting}

Условие соединения - $LEFT JOIN$. В условии $WHERE$ не исключается возможность равенства любого поля таблицы $t2$ значению $null$, значит вернуть $t1$.

\textbf{Ответ:} по таблице $t1$ будет осуществлено полное сканирование. 


\paragraph{Пример №3}
\begin{lstlisting}[language=SQL]
query = {FROM t1 LEFT JOIN t2 ON t1.a = t2.a WHERE t2.b = 5000 AND t2.c > 3 ORDER BY t2.c, t2.d}
\end{lstlisting}

Условие соединения - $LEFT JOIN$. В условии $WHERE$ исключается возможность равенства поля $t2.b$ и $t2.c$ значению $null$, значит изменим запрос на
\begin{lstlisting}[language=SQL]
query = {FROM t1 INNER JOIN t2 ON t1.a = t2.a WHERE t2.b = 5000 AND t2.c > 3 ORDER BY t2.c, t2.d}
\end{lstlisting}

Условие соединения - $INNER JOIN$. Есть сортировка $ORDER BY t2.c, \ldots$. Вернуть $t2$.

\textbf{Ответ:} по таблице $t2$ будет осуществлено полное сканирование. 