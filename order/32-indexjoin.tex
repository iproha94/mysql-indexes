\section{Indexjoin алгоритм}

\textbf{Indexjoin алгоритм} - алгоритм, для определения индексов, которые могут быть использованы СУБД для оптимизации выполнения sql запроса. Алгоритм на вход принимает join-запрос и возвращает множество пар индексов. В каждой паре первый индекс - индекс для первой таблицы, второй индекс - индекс для второй таблицы. Если в индексе нет полей, значит по таблице не нужно строить индекс. Лучшая из множества пар индексов выбирается по селективности, исходя из данных, которыми заполнены таблицы БД.

Введем некоторые обозначения:
\begin{enumerate}
\item $index(t)$ - индекс-множество для таблицы t, под которым подразумевается конечное множество индексов. Например, если $index(t) = t(a, b, c’, d, e)$, то $index(t) = {t(a, b, c, d, e), t(b, a, c, d, e), t(a, b, c, e, d), t(b, a, c, e, d)}$. Апостроф означает фиксированное поле в индексе. Нефиксированные поля можно менять местами только между фиксированными полями, началом и концом;

\item $[joinFields] \gets getJoinFields(query)$ - получить поля, по которым происходит соединение таблиц по sql-запросу $query$;

\item $index(T) \gets deleteFieldsFromIndex(index(T), [fields])$ - из $index(T)$  удалить поля, перечисленные в массиве $fields$;

\item  $[(x, y)] \gets cartesianProduct(X, Y)$ - Из пар индекс-множеств (X, Y) через прямое произведение множеств X, Y получить всевозможные пары индексов (x, y);

\item $index(T) \gets createIndexSet(query)$ - построить индекс-множество  по sql-запросу $query$.
\end{enumerate}

В качестве indexjoin алгоритма для join-запроса можно предложить алгоритм, при работе которого происходит анализ исследуемого запроса. Псевдокод представлена в алгоритме \ref{alg:indexjoin}. Блок-схема представлена на рисунке \ref{img:indexjoin}.

\begin{figure}[h]
  \centering
  \includegraphics[scale=0.5]{indexjoin.png}
  \caption{Блок-схема indexjoin алгоритма.}
  \label{img:indexjoin}
\end{figure}

\begin{algorithm}[h]
\caption{Indexjoin алгоритм}\label{alg:indexjoin}
\begin{algorithmic}[1]
 
\Function{indexjoinAlg}{query}

    \State $T \gets \Call{fullscanAlg}{query}$
    \Statex 
    
    \Comment{---begin block1---}
    \If {$T \neq null$}
        \State {$query1 \gets query$ отбросив в блоке сортировки поля, идущие после начальных полей таблицы T}
    \Else
        \State {$query1 \gets query$}
    \EndIf
    \Comment{---end block1---}
    \Statex 

    \Comment{---begin block2---}
    \State $query21 \gets query1$ без полей таблицы $t2$, при этом в условии соединения поля таблицы $t2$ заменить на $const$
    \Statex

    \State $query22 \gets query1$ без полей таблицы $t1$, при этом в условии соединения поля таблицы $t1$ заменить на $const$
    \Comment{---end block2---}
    \Statex

    \State $index(t1) \gets createIndexSet(query21)$
    \State $index(t2) \gets createIndexSet(query22)$
    \State {$joinFields \gets getJoinFields(query)$}
    \Statex

    \Comment{---begin block3---}
    \If {$T \neq null$}
        \State {$index(T) \gets deleteFieldsFromIndex(index(T), joinFields)$}
        \Statex

        \State \Return $cartesianProduct(index(t1), index(t2))$
    \EndIf
    \Comment{---end block3---}
    \Statex

    \Statex\Comment{---begin block4---}
    \State {$index1(t1) \gets index(t1)$}
    \State {$index2(t2) \gets deleteFieldsFromIndex(index(t2), joinFields)$}
    \State {$index1(t2) \gets index(t2)$}
    \State {$index2(t1) \gets deleteFieldsFromIndex(index(t1), joinFields)$}
    \Statex\Comment{---end block4---}

    \State \Return $cartesianProduct(index1(t1), index2(t2)), cartesianProduct(index1(t2), index2(t1))$
\EndFunction

\end{algorithmic}
\end{algorithm}


\paragraph{Примеры для участков кода indexjoinAlg}

$block1$: пусть $T = t2$, $query = \{\ldots ORDER BY t2.z, t2.y, t1.z, t1.y, t2.x \ldots\}$, тогда $query1 = \{\ldots ORDER BY t2.z, t2.y \ldots\}$

$block1$: пусть $T = t2$,  $query = \{\ldots ORDER BY t1.x, t2.x \ldots\}$ , тогда отбрасываются все поля (т.е. в $query1$ не будет сортировки)

$block2$: пусть $query1 = \{\ldots ON t1.a = t2.a WHERE t1.b = Z, t2.b = Y \ldots\}$, тогда $query21 = \{\ldots ON t1.a = const1 WHERE t1.b = Z \ldots\}$

$block3$: пусть $T = t1$, $index(t1) = t1(a, b)$, $index(t2) = t2(c, a)$, $query0 = \{\ldots ON t1.a = t2.a \ldots\}$, тогда $index(t1) = t1(b)$, $index(t2) = t2(c, a)$

$block4$: пусть $index(t1) = t1(a, b, c)$, $index(t2) = t2(a, b, c)$, $query1 = \{\ldots ON t1.a = t2.a \ldots\}$, тогда $ index1(t1) = t1(a, b, c)$, $index2(t2) = t2(b, c)$, $index2(t1) = t1(b, c)$, $index1(t2) = t2(a, b, c)$

\subsection{Примеры работы алгоритма}

Рассмотрим работу indexjoin алгоритма по обработке заданного SQL запроса на конкретных практических примерах.

\paragraph{Пример №1}
\begin{lstlisting}[language=SQL]
query = {t1 LEFT JOIN t2 ON t1.a = t2.a WHERE t1.b > 1 AND t2.c = 5}
\end{lstlisting}
$$ T \gets null $$
$$ query1 \gets query $$
$$ query21 \gets {ON t1.a = const WHERE t1.b > 1} $$
$$ query22 \gets {ON t2.a = const WHERE t2.c = 5} $$
$$ index(t1) \gets t1(a, b’) $$
$$ index(t2) \gets t2(a, c) $$
$$ index1(t1) \gets t1(a, b’) $$
$$ index2(t2) \gets t2(c) $$
$$ index1(t1) \gets t1(b) $$
$$ index2(t2) \gets t2(a, c) $$
$$ (t1(a, b’), t2(c)) \equiv {(t1(a, b), t2(c))} $$
$$ (t1(b), t2(a, c)) \equiv {(t1(b), t2(a, c)), (t1(b), t2(c, a))} $$

\textbf{Ответ:}  ${(t1(a, b), t2(c)), (t1(b), t2(a, c)), (t1(b), t2(c, a))}$ - множество пар индексов, наиболее подходящих данному запросу. 


\paragraph{Пример №2}
\begin{lstlisting}[language=SQL]
query = {FROM t1 LEFT JOIN t2 ON t1.a = t2.a WHERE t1.b = 5000 AND t1.c > 3 ORDER BY t2.c, t2.d}
\end{lstlisting}
$$ T \gets t1 $$
$$ query1 \gets {FROM t1 LEFT JOIN t2 ON t1.a = t2.a WHERE t1.b = 5000 AND t1.c > 3} $$
$$ query21 \gets {ON t1.a = const WHERE t1.b = 5000 AND t1.c > 3} $$
$$ query22 \gets {ON t2.a = const} $$
$$ index(t1) \gets t1(a, b, c’) $$
$$ index(t2) \gets t2(a) $$
$$ index(t1) \gets t1(b, c’) $$
$$ index(t2) \gets t2(a) $$
$$ (t1(b, c’), t2(a)) \equiv {(t1(b, c), t2(a))} $$

\textbf{Ответ:} ${(t1(b, c), t2(a))}$ - множество пар индексов, наиболее подходящих данному запросу.


\paragraph{Пример №3}
\begin{lstlisting}[language=SQL]
query = {FROM t1 LEFT JOIN t2 ON t1.a = t2.a WHERE t2.b = 5000 AND t2.c > 3 ORDER BY t2.c, t2.d}
\end{lstlisting}
$$ T \gets t2 $$
$$ query1 \gets query $$
$$ query21 \gets {ON t1.a = const} $$
$$ query22 \gets {ON t2.a = const WHERE t2.b = 5000 AND t2.c > 3 ORDER BY t2.c, t2.d} $$
$$ index(t1) \gets t1(a) $$
$$ index(t2) \gets t2(a, b, c’, d’) $$
$$ index(t1) \gets t1(a) $$
$$ index(t2) \gets t2(b, c', d’) $$
$$ (t1(a), t2(b, c’, d’)) \equiv {(t1(a), t2(b, c, d))} $$

\textbf{Ответ:} ${(t1(a), t2(b, c, d))}$ - множество пар индексов, наиболее подходящих данному запросу.


\paragraph{Пример №4}
\begin{lstlisting}[language=SQL]
query = {FROM t1 LEFT JOIN t2 ON t2.a = t1.a ORDER BY t2.b}
\end{lstlisting}
$$ T \gets t1 $$
$$ query1 \gets {t1 LEFT JOIN t2 ON t2.a = t1.a} $$
$$ query21 \gets {ON t1.a = const} $$
$$ query22 \gets {ON t2.a = const} $$
$$ index(t1) \gets t1(a) $$
$$ index(t2) \gets t2(a) $$
$$ index(t1) \gets t1() $$
$$ index(t2) \gets t2(a) $$
$$ (t1(), t2(a)) \equiv {(t1(), t2(a))} $$

\textbf{Ответ:} ${(t1(), t2(a))}$ - множество пар индексов, наиболее подходящих данному запросу.

\paragraph{Пример №5}
\begin{lstlisting}[language=SQL]
query = {
    t1 INNER JOIN t2
        ON t1.a = t2.a
    WHERE t1.b = const
    ORDER BY t1.c
}
\end{lstlisting}

\textbf{Ответ:} ${(t1(b, c), t2(a))}$ - множество пар индексов, наиболее подходящих данному запросу.

% \paragraph{Пример №6}
% \begin{lstlisting}[language=SQL]
% query = {
%     t1 LEFT JOIN t2
%         ON t2.id = t1.id_n
%             AND t2.c = 5
%     WHERE t1.b > 1
% }
% \end{lstlisting}

% \textbf{Ответ:} ${}$ - множество пар индексов, наиболее подходящих данному запросу.

% \paragraph{Пример №7}
% \begin{lstlisting}[language=SQL]
% query = {
%     t1 LEFT JOIN t2
%         ON t2.a = t1.a
%     ORDER BY t1.b
% }
% \end{lstlisting}

% \textbf{Ответ:} ${}$ - множество пар индексов, наиболее подходящих данному запросу.

% \paragraph{Пример №7}
% \begin{lstlisting}[language=SQL]
% query = {
%     SELECT *
%     FROM t1 LEFT JOIN t2
%         ON t1.a = t2.a
%     WHERE t1.b = 5000
%     ORDER BY t2.a
% }
% \end{lstlisting}

% \textbf{Ответ:} ${}$ - множество пар индексов, наиболее подходящих данному запросу.

% \paragraph{Пример №8}
% \begin{lstlisting}[language=SQL]
% query = {
%     SELECT *
%     FROM t1 LEFT JOIN t2
%         ON t1.b = t2.b
%     WHERE t1.c = 5000
%         AND t2.c = 5000
% }
% \end{lstlisting}

% \textbf{Ответ:} ${}$ - множество пар индексов, наиболее подходящих данному запросу.

% \paragraph{Пример №9}
% \begin{lstlisting}[language=SQL]
% query = {
%     SELECT *
%     FROM t1 INNER JOIN t2
%         ON t1.b = t2.b
%             AND t2.c = 5000
%     WHERE t1.c > 0
% }
% \end{lstlisting}

% \textbf{Ответ:} ${}$ - множество пар индексов, наиболее подходящих данному запросу.


% \paragraph{Пример №10}
% \begin{lstlisting}[language=SQL]
% query = {
%     SELECT t2.b
%     FROM t1 LEFT JOIN t2
%         ON t1.c = t2.c
%             AND t2.d = 5000
%     ORDER BY t2.b
% }
% \end{lstlisting}

% \textbf{Ответ:} ${}$ - множество пар индексов, наиболее подходящих данному запросу.


% \paragraph{Пример №11}
% \begin{lstlisting}[language=SQL]
% query = {
%     FROM t1 LEFT JOIN t2
%         ON t1.b = t2.b
%     WHERE t1.c = 5000
%         AND t2.c = 5000
%     ORDER BY t2.b
% }
% \end{lstlisting}

% \textbf{Ответ:} ${}$ - множество пар индексов, наиболее подходящих данному запросу.


% \paragraph{Пример №12}
% \begin{lstlisting}[language=SQL]
% query = {
%     SELECT *
%     FROM t1 LEFT JOIN t2
%         ON t1.b = t2.b
%     WHERE t2.d like 'ABC%'
%     ORDER BY t2.d
% }
% \end{lstlisting}

% \textbf{Ответ:} ${}$ - множество пар индексов, наиболее подходящих данному запросу.


% \paragraph{Пример №13}
% \begin{lstlisting}[language=SQL]
% query = {
%     SELECT *
%     FROM t1 LEFT JOIN t2
%         ON t1.b = t2.b
%             AND t2.a like 'A%'
%     ORDER BY t2.d
% }
% \end{lstlisting}

% \textbf{Ответ:} ${}$ - множество пар индексов, наиболее подходящих данному запросу.