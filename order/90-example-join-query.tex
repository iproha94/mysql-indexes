\paragraph{Приложение А. Примеры получения индексов для сложных запросов}

Рассмотрим получение индексов для заданного SQL запроса на конкретных примерах.

\paragraph{Пример №1}

\begin{lstlisting}[language=SQL]
query := t1 LEFT JOIN t2 ON t1.a = t2.a WHERE t1.b > 1 AND t2.c = 5

T := NULL 

query1 := query 

query21 := ON t1.a = const WHERE t1.b > 1 
query22 := ON t2.a = const WHERE t2.c = 5

index(t1) := t1(a, b!) 
index(t2) := t2(a, c) 

index1(t1) := t1(a, b!) 
index2(t2) := t2(c) 

index1(t2) := t2(a, c) 
index2(t1) := t1(b) 
\end{lstlisting}

(t1(a, b!), t2(c)) $\equiv$ (t1(a, b), t2(c)) 

(t1(b), t2(a, c)) $\equiv$ (t1(b), t2(a, c)), (t1(b), t2(c, a)) 

\textbf{Ответ:} \textit{(t1(a, b), t2(c)), (t1(b), t2(a, c)), (t1(b), t2(c, a))} - множество пар индексов, наиболее подходящих данному запросу. 


\paragraph{Пример №2}
\begin{lstlisting}[language=SQL]
query := FROM t1 LEFT JOIN t2 ON t1.a = t2.a WHERE t1.b = 5000 AND t1.c > 3 ORDER BY t2.c, t2.d

T := t1 

query1 := FROM t1 LEFT JOIN t2 ON t1.a = t2.a WHERE t1.b = 5000 AND t1.c > 3

query21 := ON t1.a = const WHERE t1.b = 5000 AND t1.c > 3 
query22 := ON t2.a = const

index(t1) := t1(a, b, c!) 
index(t2) := t2(a) 

index(t1) := t1(b, c!) 
\end{lstlisting}
 
(t1(b, c!), t2(a)) $\equiv$ {(t1(b, c), t2(a))} 

\textbf{Ответ:} \textit{(t1(b, c), t2(a))} - множество пар индексов, наиболее подходящих данному запросу.


\paragraph{Пример №3}
\begin{lstlisting}[language=SQL]
query := FROM t1 LEFT JOIN t2 ON t1.a = t2.a WHERE t2.b = 5000 AND t2.c > 3 ORDER BY t2.c, t2.d

T := t2

query1 := query 

query21 := ON t1.a = const 
query22 := ON t2.a = const WHERE t2.b = 5000 AND t2.c > 3 ORDER BY t2.c, t2.d 

index(t1) := t1(a) 
index(t2) := t2(a, b, c!, d!) 

index(t2) := t2(b, c!, d!) 
\end{lstlisting}

(t1(a), t2(b, c!, d!)) $\equiv$ (t1(a), t2(b, c, d)) 

\textbf{Ответ:} \textit{(t1(a), t2(b, c, d))} - множество пар индексов, наиболее подходящих данному запросу.


\paragraph{Пример №4}
\begin{lstlisting}[language=SQL]
query := FROM t1 LEFT JOIN t2 ON t2.a = t1.a ORDER BY t2.b

T := t1 

query1 := t1 LEFT JOIN t2 ON t2.a = t1.a 

query21 := ON t1.a = const 
query22 := ON t2.a = const 

index(t1) := t1(a) 
index(t2) := t2(a) 

index(t1) := t1() 
\end{lstlisting}
 
(t1(), t2(a)) $\equiv$ (t1(), t2(a)) 

\textbf{Ответ:} \textit{(t1(), t2(a))} - множество пар индексов, наиболее подходящих данному запросу.


\paragraph{Пример №5}
\begin{lstlisting}[language=SQL]
query := 
    t1 INNER JOIN t2
        ON t1.a = t2.a
    WHERE t1.b = const
    ORDER BY t1.c
\end{lstlisting}

\textbf{Ответ:} \textit{(t1(b, c), t2(a))} - множество пар индексов, наиболее подходящих данному запросу.


% \paragraph{Пример №6}
% \begin{lstlisting}[language=SQL]
% query = {
%     t1 LEFT JOIN t2
%         ON t2.id = t1.id_n
%             AND t2.c = 5
%     WHERE t1.b > 1
% }
% \end{lstlisting}

% \textbf{Ответ:} ${}$ - множество пар индексов, наиболее подходящих данному запросу.

% \paragraph{Пример №7}
% \begin{lstlisting}[language=SQL]
% query = {
%     t1 LEFT JOIN t2
%         ON t2.a = t1.a
%     ORDER BY t1.b
% }
% \end{lstlisting}

% \textbf{Ответ:} ${}$ - множество пар индексов, наиболее подходящих данному запросу.

% \paragraph{Пример №7}
% \begin{lstlisting}[language=SQL]
% query = {
%     SELECT *
%     FROM t1 LEFT JOIN t2
%         ON t1.a = t2.a
%     WHERE t1.b = 5000
%     ORDER BY t2.a
% }
% \end{lstlisting}

% \textbf{Ответ:} ${}$ - множество пар индексов, наиболее подходящих данному запросу.

% \paragraph{Пример №8}
% \begin{lstlisting}[language=SQL]
% query = {
%     SELECT *
%     FROM t1 LEFT JOIN t2
%         ON t1.b = t2.b
%     WHERE t1.c = 5000
%         AND t2.c = 5000
% }
% \end{lstlisting}

% \textbf{Ответ:} ${}$ - множество пар индексов, наиболее подходящих данному запросу.

% \paragraph{Пример №9}
% \begin{lstlisting}[language=SQL]
% query = {
%     SELECT *
%     FROM t1 INNER JOIN t2
%         ON t1.b = t2.b
%             AND t2.c = 5000
%     WHERE t1.c > 0
% }
% \end{lstlisting}

% \textbf{Ответ:} ${}$ - множество пар индексов, наиболее подходящих данному запросу.


% \paragraph{Пример №10}
% \begin{lstlisting}[language=SQL]
% query = {
%     SELECT t2.b
%     FROM t1 LEFT JOIN t2
%         ON t1.c = t2.c
%             AND t2.d = 5000
%     ORDER BY t2.b
% }
% \end{lstlisting}

% \textbf{Ответ:} ${}$ - множество пар индексов, наиболее подходящих данному запросу.


% \paragraph{Пример №11}
% \begin{lstlisting}[language=SQL]
% query = {
%     FROM t1 LEFT JOIN t2
%         ON t1.b = t2.b
%     WHERE t1.c = 5000
%         AND t2.c = 5000
%     ORDER BY t2.b
% }
% \end{lstlisting}

% \textbf{Ответ:} ${}$ - множество пар индексов, наиболее подходящих данному запросу.


% \paragraph{Пример №12}
% \begin{lstlisting}[language=SQL]
% query = {
%     SELECT *
%     FROM t1 LEFT JOIN t2
%         ON t1.b = t2.b
%     WHERE t2.d like 'ABC%'
%     ORDER BY t2.d
% }
% \end{lstlisting}

% \textbf{Ответ:} ${}$ - множество пар индексов, наиболее подходящих данному запросу.


% \paragraph{Пример №13}
% \begin{lstlisting}[language=SQL]
% query = {
%     SELECT *
%     FROM t1 LEFT JOIN t2
%         ON t1.b = t2.b
%             AND t2.a like 'A%'
%     ORDER BY t2.d
% }
% \end{lstlisting}

% \textbf{Ответ:} ${}$ - множество пар индексов, наиболее подходящих данному запросу.